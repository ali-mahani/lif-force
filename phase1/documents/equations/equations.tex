\documentclass[12pt, letter]{article}

\usepackage[top=2.5cm,right=2cm,left=2cm,bottom=2.5cm]{geometry}
\usepackage[]{enumitem}
\usepackage{hyperref}
\usepackage{graphicx}
\usepackage{amsmath,amssymb}
\usepackage{physics}

\author{Ali Mahani}
\title{Morris Lecar Equations to use for FORCE training}
%\date{}

\begin{document}
	\maketitle
	
	The equations from the lecture series for the Morris Lecar model are as follows:
	\begin{equation}
		\begin{aligned}
			C\dv{V_i}{t} &= I - g_L(V_i - E_L) = g_K n_i (V_i - E_K) - g_{Ca} m_\infty(V_i) (V_i - E_{Ca}) + I_{psp, i}\\
			I_{psp, i} &= -\sum_{j=1}^{N} \bar{g}_{ij} s_j(t) (V_i - E_{ij})\\
			\dv{n_i}{t} &= \left[\frac{n_\infty(V_i) - n_i}{\tau_n(V_i)}\right]\\
			m_\infty(V_i) &= \frac{1}{2} \left[1 + \tanh(\frac{V_i - V_1}{V_2})\right]\\
			n_\infty(V_i) &= \frac{1}{2} \left[1 + \tanh(\frac{V_i - V_3}{V_4})\right]\\
			\tau_n(V_i) &= \left[\phi \cosh(\frac{V_i - V_3}{2 V_4})\right]^{-1}\\
			\dv{s_i}{t} &= a_r T(V_i)(1 - s_i) - a_d s_i\\
			T(V_i) &= \frac{T_{max}}{1 + \exp(-\frac{V_i - V_T}{K_p})}					
		\end{aligned}
	\end{equation}
	where 
	\begin{equation}
		E_{ij} = \{E_+, E_-\}
	\end{equation}
	for $E_+$ for AMPA and $E_-$ for GABA.
	and 
	\begin{equation}
		E_{ij} = E_{kj}, \qquad \forall i,j,k \le N
	\end{equation}
	where the following \textbf{variables}
	\begin{itemize}
		\item $V_i$: The membrane potential of the neurons.
		\item $n_i$: recovery variable: probability that the $K^+$ channel is conducting. 
	\end{itemize}
	and \textbf{parameters} in the \textbf{Hodgkin-Class I} with random coupling:
	\begin{itemize}
		\item $I = 100 pA$: The applied current.
		\item $C = 20 pF$: membrane capacitance.
		\item $g_L = 2, g_K = 8, g_{Ca} = 4 nS$: leak, K, Ca conductances through the memeber ion channels.
		\item $E_L = -60, E_K = -84, E_{Ca} = 120 mV$: equilibrium potential of relevant ion channels.
		\item $V_1=-1.2, V_2 = 18, V_3 = 12, V_4 = 17.4 mV$: tuning parameters for steady state and time constant.
		\item $\phi = 0.067 Hz$: the reference frequency.
		\item $a_r = 1.1, a_d = 0.19 ms^{-1}$: rise and decay inverse time constants for the post-synaptic weights.
		\item $V_T = 2, K_p = 5 mV, T_{max} = 1$: Tuning parameters for the post-synaptic potential.
		\item $\bar{g}_{ij}$: weight matrix taken randomly from a normal distribution. Details to follow.
	\end{itemize}
	
	In vector form, the post-synaptic potential becomes
	\begin{equation}
		\vec{I}_{psp} = \left(\mathbf{\bar{g}}\ + \mathbf{\hat{\eta}}.\mathbf{\hat{\phi}}^\intercal\right) \otimes (\vec{V}.\mathbf{1}^\intercal - \mathbf{\hat{E}}).\vec{s}(t)
	\end{equation}
	
\end{document}